\documentclass[12pt,hyperref={colorlinks,citecolor=blue,urlcolor=peking_blue,linkcolor=},aspectratio=169]{beamer}
\usepackage{PekingU}
\usefonttheme{serif}
\usepackage{lipsum}
%\usepackage[scheme = plain]{ctex}
\usepackage{hyperref}
\usepackage{charter} % Nicer fonts
% other packages
\usepackage{latexsym,amsmath}
\usepackage{amssymb}
\usepackage{graphicx}
\usepackage{bm}
\usepackage{natbib}
\usepackage{wrapfig}
\usepackage{amsfonts} 
\usepackage{ragged2e}
\usepackage{parskip}
\usepackage{multicol}
\usepackage{calligra}

\apptocmd{\frame}{}{\justifying}{} % Allow optional arguments after frame.

\newcommand{\theHalgorithm}{\arabic{algorithm}}
\theoremstyle{plain}
\newtheorem{axiom}{Axiom}
\newtheorem{claim}[axiom]{Claim}
\newtheorem{assumption}{Assumption}
\newtheorem{remark}{Remark}
\newtheorem{proposition}{Proposition}
\setbeamertemplate{theorems}[numbered] 


% change for your title page information
\author[Alauddin Maulana Hirzan]{Alauddin Maulana Hirzan, S.Kom., M.Kom.\\ NIDN. 0607069401}
\title{Mobile Programming}
\subtitle{Pertemuan 01}
\institute{Fakultas Teknologi Informasi dan Komunikasi, Universitas Semarang}
\date{}

\newif\ifplacelogo % create a new conditional
\placelogotrue % set it to true
\logo{
\ifplacelogo
\includegraphics[width=2cm]{Figures/ftik-usm}
\fi
}
% official colors match with the PKU red
\def\cmd#1{\texttt{\color{red}\footnotesize $\backslash$#1}}
\def\env#1{\texttt{\color{blue}\footnotesize #1}}
\definecolor{deepblue}{rgb}{0,0,0.5}
\definecolor{deepred}{rgb}{0.6,0,0}
\definecolor{deepgreen}{rgb}{0,0.5,0}
\definecolor{halfgray}{gray}{0.55}

\begin{document}
{
\setbeamertemplate{logo}{}
\begin{frame}
    \titlepage
    \begin{figure}[htpb]
        \begin{center}
            \includegraphics[width=0.3\linewidth]{Figures/ftik-usm}
        \end{center}
    \end{figure}
\end{frame}
}

\placelogofalse

\section{Biodata Singkat}
\begin{frame}{Biodata Singkat}
	\framesubtitle{Tak Kenal Maka Tak Sayang}
	\begin{columns}
		\begin{column}{.2\textwidth}
			\begin{figure}[htpb]
				\begin{center}
					\includegraphics[width=\textwidth]{Figures/profile}
				\end{center}
			\end{figure}
		\end{column}
		\begin{column}{.8\textwidth}
			\begin{enumerate}
				\justifying
				\item Nama : Alauddin Maulana Hirzan, S.Kom., M.Kom.
				\item NIDN : 0607069401
				\item NIS : 06557003102238
				\item E-Mail : \hyperlink{mailto:maulanahirzan@usm.ac.id}{maulanahirzan@usm.ac.id}
				\item Web Profile : \hyperlink{http://maulanahirzan.pythonanywhere.com/}{Portofolio} | \hyperlink{https://ftik.usm.ac.id/profile-alauddin-maulana-hirzan-s-kom-m-kom/}{Resume}
			\end{enumerate}
		\end{column}
	\end{columns}
	\vfill
	\begin{block}{Fokus Kajian}
		\begin{itemize}
			\item Kajian Jaringan dan Internet of Things (IoT)
			\item Kajian Web dan Mobile
		\end{itemize}
	\end{block}
\end{frame}

\section{Kontrak Perkuliahan}
\begin{frame}{Kontrak Kuliah}
	\framesubtitle{Peraturan Dasar Perkuliahan}
	Mahasiswa dan Dosen memiliki kesepakatan berupa:
	\vfill
	\begin{columns}
		\begin{column}{.5\textwidth}
			\begin{itemize}
				\item \textbf{Mahasiswa}
				\begin{enumerate}
					\item Datang tepat waktu
					\item Mengerjakan tugas dan kuis
					\item Mengikuti ujian
					\item Mengikuti perkuliahan dengan kondusif
				\end{enumerate}
			\end{itemize}
		\end{column}
		\begin{column}{.5\textwidth}
			\begin{itemize}
				\item \textbf{Dosen}
				\begin{enumerate}
					\item Memberikan materi dan nilai
					\item Mengumumkan kelas apabila daring
					\item Memberikan tugas dan kuis
					\item Memberikan ujian atau proyek
				\end{enumerate}
			\end{itemize}
		\end{column}
	\end{columns}
\end{frame}

\section{Deskripsi dan Capaian Mata Kuliah}
\begin{frame}{Deskripsi Mata Kuliah}
	\justifying
	Mata Kuliah \textbf{\textit{Mobile Programming}} ini mempelajari pengetahuan tentang teori dan dasar pembuatan aplikasi bergerak Android dan beberapa tool/IDE untuk membuat aplikasi bergerak berbasis Android tersebut. \\~\\
	Mata kuliah ini melatih keterampilan mahasiswa dalam membuat program berbasis  mobile untuk aplikasi stand alone, 
\end{frame}

\begin{frame}{Capaian Mata Kuliah}
	\framesubtitle{Capaian yang didapatkan Mahasiswa}
	\justifying
	Mata kuliah ini memiliki Capaian Berupa:
	\vfill
	\begin{enumerate}
		\justifying
		\item Menguasai konsep teoritis teknologi Android
		\item Mampu memiliki pengetahuan mengenai Layout Desain dan Widget View
		\item Mampu untuk membangun aplikasi sederhana
		\item Mampu memahami konseptual, rancangan dan implementasi sistem basis data
	\end{enumerate}
\end{frame}

\section{Rumusan Penilaian}
\begin{frame}{Penilaian Mata Kuliah}
	Mata kuliah ini dinilai dengan porsi sebagai berikut:
	\vfill
	\begin{table}[]
		\begin{tabular}{clcc}
			\hline
			No & \multicolumn{1}{c}{Kategori} & Proporsi (\%) & Keterangan                                                   \\ \hline
			1  & Presentasi                   & 10\%          & \multicolumn{1}{l}{75\% (Pagi) | 50\% (Sore)} \\
			2  & Tugas                        & 20\%          & \multicolumn{1}{l}{2 Tugas 2 Kuis}                           \\
			3  & Ujian Tengah Semester        & 35\%          &                                                              \\
			4  & Ujian Akhir Semester         & 35\%          &                                                              \\ \hline
			\multicolumn{2}{c}{Total}         & 100\%         &                                                              \\ \hline
		\end{tabular}
	\end{table}	
\end{frame}

\section{Pokok Bahasan}
\begin{frame}{Pokok Bahasan}
	\framesubtitle{Apa yang akan di bahas?}
	\justifying
	\begin{multicols}{2}
		\begin{enumerate}
			\justifying
			\item Pengenalan Mobile Programming
			\item Aktivitas dan Tujuan Mobile Programing
			\item Test, men-debug, dan kompatibilitas mundur
			\item Interaksi pengguna dan navigasi intuitif
			\item Pengalaman pengguna  
			\item Pengujian User Interface
			\item Tugas dari Latar Layar Mobile
			\item Menyimpan data dengan SQLite
		\end{enumerate}
	\end{multicols}
\end{frame}

\begin{frame}
	\begin{center}
		{\fontsize{40pt}{46pt}\calligra\color{peking_dark} Terima Kasih}
	\end{center}
\end{frame}

\end{document}